\documentclass{report}

%Packages
%-------------------------------------------------------------------------------
	\usepackage[T1]{fontenc}
	\usepackage[a4paper, total={6in, 8in}]{geometry}%Margins
	\usepackage{polski}
	\usepackage[utf8]{inputenc}
	\usepackage[polish]{babel}
	\usepackage{graphicx}
	\usepackage{fix-cm}
	\usepackage{tocloft}%Dots in table of contents
	\usepackage{indentfirst}%Indent at the begining, after section/subsection
	\usepackage{fancyhdr}
%-------------------------------------------------------------------------------
%end Packages



%Headers and footers
\pagestyle{fancy}
\chead{\Huge{\uppercase{POLITECHNIKA ŚLĄSKA}}}
\lfoot{Gliwice, rok akadem. 2018/2019}
%end Headers and footers



%Metadata
%-------------------------------------------------------------------------------
\title{Porównanie wybranych algorytmów heurystycznych w rozwiązywaniu zagadnień odwrotnych}
\author{Kamil Kryus}
%-------------------------------------------------------------------------------
%end Metadata

%Renew
%-------------------------------------------------------------------------------
%Dots in table of contents
\renewcommand \cftchapdotsep{4.5}
%-------------------------------------------------------------------------------
%end Renew

%Custom commands
%-------------------------------------------------------------------------------
\newcommand{\hugeFontSize}{}
\newcommand{\newLine}{~\\}
\newcommand{\si}{ś}
%-------------------------------------------------------------------------------
%end Custom commands


\makeatletter
\begin{document}
	\begin{titlepage}
	\thispagestyle{fancy}
		\begin{center}
			\newLine
			\includegraphics[width=0.4\textwidth]{logo}\\ \newLine
			\huge{Wydział Matematyki Stosowanej} \\ \newLine

			PRACA INŻYNIERSKA \\ \newLine
		\end{center}
			\LARGE{Temat: \@title} \\ \newLine
			Dyplomant: \@author \\
			Kierunek studiów: Informatyka \\
			Specjalizacja: Programowanie Aplikacji Mobilnych \\ \newLine
			Opiekun pracy: dr inż. Adam Zielonka \\
	\end{titlepage}
	\newpage
%Table of Contents
	\tableofcontents
%end Table of Contents
	\newpage
%Wprowadzenie
	\chapter{Wprowadzenie}
	
%end Wprowadzenie
%Opis problemu
	\chapter{Opis problemu}

%end Opis problemu
%Opis algorytmów
	\chapter{Opis algorytmów}
		\section{Algorytm symulowanego wyżarzania}
			\subsection{Opis}
	Algorytm ten został stworzony wzorując się na zjawisku wyżarzania w metalurgii, które polega na nagrzaniu elementu stalowego do odpowiedniej temperatury, przetrzymaniu go w tej temperaturze przez pewien czas, a następnie powolnym jego schłodzeniu. Sam algorytm natomiast bazuje na metodach Monte-Carlo i w pewnym sensie może być rozważany jako algorytm iteracyjny.\\ \newLine
Główną istotą i zarazem zaletą tego algorytmu jest wykonywanie pewnych losowych przeskoków do sąsiednich rozwiązań, dzięki czemu jest w stanie uniknąć wpadania w lokalne minimum. Algorytm ten najczę\si ciej jest używany do rozwiązywania problemów kombinatorycznych, jak np. problemu komiwojażera.
			\subsection{Parametry}

			\subsection{Kroki algorytmu}


%end Opis algorytmów
%Cel
	\chapter{Cel}
	Przetestowanie algorytmu wyżarzania, bla bla
%end Cel
%Funkcje testowe
	\chapter{Funkcje testowe}
		\section{x\^2+y\^2}
		\section{Funkcja Rastrigina}
		\section{Funkcja Rosenbrocka}
%end Funkcje testowe
%Dobór parametrów
	\chapter{Dobór parametrów}

%end Dobór parametrów
%Implementacja
	\chapter{Implementacja}

%end Implementacja
%Dostosowanie algorytmów do funkcji testowej zadań odwrotnych
	\chapter{Dostosowanie algorytmów do funkcji testowej zadań odwrotnych}

%end Dostoswanie algorytmów do funkcji testowej zadań odwrotnych
%Narzędzia i technologie
	\chapter{Narzędzia i technologie}
		\section{Użyte narzędzia}

		\section{Użyte technologie}
%end Narzędzia i technologie
%Podsumowanie
	\chapter{Podsumowanie}
%end %Podsumowanie
		\section{Dalsze kierunki rozwoju}
	Foobar

\end{document}