\documentclass[twoside]{projektInzynierskiMS1}
\usepackage{polski}
\usepackage[utf8]{inputenc}
\usepackage{amsmath}
\usepackage{graphicx}%obrazki
\usepackage{array,etoolbox}%indeksowanie wierszy w tabelach
\preto\tabular{\setcounter{magicrownumbers}{0}}
\newcounter{magicrownumbers}
\newcommand\rownumber{\stepcounter{magicrownumbers}\arabic{magicrownumbers}}
\usepackage{multirow,tabularx}
\usepackage{etoolbox}
\usepackage{float}%obrazki na tej samej stronie
\newcounter{rowcnt}
\newcommand\rownum{\ifnumequal{\value{rowcnt}}{0}{\textbf{Nr.}}{\therowcnt.}\refstepcounter{rowcnt}}
\AtEndEnvironment{tabularx}{\setcounter{rowcnt}{0}}



%\drukJednostronny

%% tytuł promotor iautor (\title to komenda standardowa)
\title{Porównanie wybranych algorytmów heurystycznych w rozwiązywaniu zagadnień odwrotnych}
\promotor{dr inż. Adam Zielonka}


%% każdy autor musi mieć 3 argumenty: imię nazwisko, nr albumu, opis wkładu
\autor{Kamil Kryus}{246591}
	


%Custom commands
%-------------------------------------------------------------------------------
\newcommand{\hugeFontSize}{}
\newcommand{\newLine}{~\\}
\newcommand{\si}{ś}
\newcommand{\SI}{Ś}
%-------------------------------------------------------------------------------
%end Custom commands




%\NumeryNaPoczatku
%% numeracja wzorów tu włączona typu (1.2.3), ta druga to typu (1.2), domyślnie typu (1)
%\subsectionWzory
% \sectionWzory  

%\rozdzialy


%\literowaNumeracjaDodatkow %% włączy numerację dodatków literami
%\rzymskaNumeracjaDodatkow  %%włączy numerację dodatków liczbami rzymskimi

%% wyłączenie wyjaśnień:
\bezWyjasnien

%% standardowe komendy \newtheorem  działają jak woryginale
\newtheorem{tw}{Twierdzenie}%[subsection]
\newtheorem{twa}{Twierdzenie}%[section]
\newtheorem{dd}{Definicja}%[subsection]

\begin{document}

Na problem znalezienia optymalnego rozwiązania możemy trafić w wielu dziedzinach życia i nauki, np. w matematyce szukając globalnego minimum/maximum lub szukając najkrótszego połączenia pomiędzy kilkoma miastami (problem komiwojażera). Szukając rozwiązań, zawsze chcemy by było ono jak najlepsze (lub dokładnie najlepsze) i zostało znalezione w rozsądnym czasie. W tym celu wła\si nie jest używana heurystyka.\\ 


Metody heurystyczne znacznie skracają czas wyszukiwania rozwiązania problemu, aczkolwiek często ich wynikiem jest jedynie wynik bardzo zbliżony do najlepszego. Pozwala jednak nam to na jedną z dwóch rzeczy:
\begin{enumerate}
	\item zaakceptowanie takiego wyniku, gdy dokładne rozwiązanie nie jest konieczne (np. kompresja obrazu),
	\item zawężenia zakresu i dalszych poszukiwań najlepszego rozwiązania. \\
\end{enumerate}
Jednak aby metodę heurystyczną uznać za dobrą, musi ona spełniać 3 wymagania:
\begin{itemize}
	\item[--] rozwiązanie jest możliwe do znalezienia z rozsądnym wysiłkiem obliczeniowym,
	\item[--] rozwiązanie powinno być bliskie optymalnemu,
	\item[--] prawdopodobieństwo uzyskania złego rozwiązania powinno być niskie.
\end{itemize}

\section{Geneza i cel projektu}

\subsection{Cel}
	Przetestowanie algorytmu wyżarzania, bla bla 
pomimo, ze najlepiej sprawdza sie w zadaniach kombinatorycznych, to jak bedzie dzialac przy szukaniu minimum globalnym i zadaniach odwrotnych


\section{Opis algorytmów}
	\subsection{Algorytm symulowanego wyżarzania}
				Algorytm ten został stworzony wzorując się na zjawisku wyżarzania w metalurgii, które polega na nagrzaniu elementu stalowego do odpowiedniej temperatury, przetrzymaniu go w tej temperaturze przez pewien czas, a następnie powolnym jego schłodzeniu. Sam algorytm natomiast bazuje na metodach Monte-Carlo i w pewnym sensie może być rozważany jako algorytm iteracyjny.\\ \newLine
Główną istotą i zarazem zaletą tego algorytmu jest wykonywanie pewnych losowych przeskoków do sąsiednich rozwiązań, dzięki czemu jest w stanie uniknąć wpadania w lokalne minimum. Algorytm ten najczę\si ciej jest używany do rozwiązywania problemów kombinatorycznych, jak np. problemu komiwojażera.
		\subsubsection{Parametry}
		


\noindent \underline{Początkowa konfiguracja} \\ \newLine
\indent W tym kroku powinniśmy zainicjalizować naszą temperaturę wysoką wartością oraz znaleźć początkowe losowe rozwiązanie naszego problemu. 
\\ \newLine

\noindent \underline{Temperatura} \\ \newLine
\indent Temperatura jest zarówno czynnikiem iteracyjnym, jak i jest związana z funkcją prawdopodobieństwa zamiany gorszego rozwiązania na lepsze. Zatem zakres temperatury powinien być taki, aby na początku działania naszego algorytmu dawał wysoką możliwość zamian, a wraz z postępem iteracji te prawdopodobieństwo się zmniejszało i pod koniec było bliskie zeru.\\ \newLine


\noindent \underline{Końcowa temperatura} \\ \newLine
\indent Jest to bardzo mała wartość. Temperatura osiągając taki poziom informuje nas iż proces wyżarzania się zakończył i rozwiązanie zostało znalezione.
Wartość ta powinna być na tyle mała, by temperatura będąc niewiele większa prowadziła do bardzo niskiego prawdopodobieństwa, a jednocze\si nie nie wymagało to zbyt dużej ilo\si ci iteracji. \\ \newLine


\noindent \underline{Powtarzanie pewną ilość razy dla zadanej temperatury} \\ \newLine
\indent Wartość ta powinna być z góry ustalona i powinna dać nam możliwość sprawdzenia wielu sąsiadów obecnego rozwiązania, równocześnie nie powodując zbyt dużego obciążenia dla algorytmu.\\ \newLine


\noindent \underline{Znajdowanie losowego sąsiada poprzedniego rozwiązania} \\ \newLine
\indent Funkcja ta powinna nam pozwalać przejrzeć jak najszerszy zakres rozwiązań, a jednocze\si nie pozwolić na przeszukiwanie coraz to bliższych sąsiadów obecnie najlepszego rozwiązania, zatem warto uzależnić tą funkcję od stopnia ukończenia globalnych iteracji.\\ \newLine


\noindent \underline{Funkcja kosztu} \\ \newLine
\indent Poprzez funkcję kosztu rozumiemy różnicę pomiędzy obecnie najlepszym rozwiązaniem, a nowym. Przy poszukiwaniu globalnego minimum warto\si ć większa jest gorszym rozwiązaniem, dzięki czemu wynikiem tej funkcji jest zawsze liczba ujemna.
 W tym wypadku nowe rozwiązanie zawsze jest gorsze (a tym samym przy poszukiwaniu minimum globalnego posiada warto\si ć większą), to w tym wypadku warto\si ć jest ujemna.\\ \newLine


\noindent \underline{Prawdopodobieństwo zamiany P} \\ \newLine
\indent Prawdopodobieństwo jest potrzebne do decyzji czy zamienić nasze nowe i gorsze rozwiązanie z wcze\si niejszym, lepszym. 

Prawdopodobieństwo zależy od funkcji kosztu oraz obecnej temperatury. Prawdopodobieństwo zatem można przedstawić w następujący sposób:
\[P = e^\frac{\Delta E}{T} \]
gdzie:\\
\begin{flalign}
& \Delta E - funkcja\ kosztu& 
\end{flalign}
T - obecna warto\si ć temperatury\\

Prawdopodobieństwo to wraz ze spadkiem warto\si ci funkcji kosztu maleje (gdyż jest zawsze ujemne), natomiast wyższa warto\si ć temperatury zwiększa to prawdopodobieństwo. Decydując o tym czy powinniśmy zamienić nasze gorsze rozwiązanie z lepszym powinniśmy porównać obliczone prawdopodobieństwo z wartością losową zawierającą się w zakresie [0,1].\\ \newLine


\noindent \underline{Zmniejszanie temperatury} \\ \newLine
\indent Szybkość zmniejszania nie powinna być zbyt duża, aby pozwolić algorytmowi na sprawdzenie jak największego zakresu możliwych rozwiązań, a jednocześnie niezbyt wolna, gdyż może to spowodować zbyt wolny spadek prawdopodobieństwa, a tym samym finalnie pozwolić na pozostanie w gorszym rozwiązaniu. W większości opracowań można spotkać ten proces, jako mnożnik temperatury w zakresie [0.8;0.99].\\ \newLine
		
		\subsubsection{Kroki algorytmu}
		
		Algorytm ten można również przedstawić za pomocą listy kroków:

\begin{enumerate}
	\item Zainicjalizuj początkową konfigurację
	\item Dopóki temperatura > minimum, powtarzaj:
	\begin{enumerate}
		\item Powtórz pewną ilość razy dla danej temperatury
		\begin{enumerate}
			\item Znajdź losowo sąsiada poprzedniego rozwiązania
			\item Sprawdź czy rozwiązanie jest lepsze od poprzedniego (funkcja kosztu)
			\begin{enumerate}
				\item Jeżeli jest, zamień rozwiązania
				\item Jeżeli nie jest, zamień rozwiązania z pewnym prawdopodobieństwem P
			\end{enumerate}
		\end{enumerate}
		\item Zmniejsz temperaturę
	\end{enumerate}
\end{enumerate}
		

\section{Funkcje testowe}
	\subsection{Funkcja kwadratowa dwóch parametrów}
	Jako pierwszą funkcję do testów przyjąłem funkcję kwadratową dwóch parametrów następującej postaci:
\[f(x, y) = x^2 + y^2 \] \\

Funkcja ta jest funkcją parzystą i przyjmuje tylko warto\si ci nieujemne. Na potrzeby projektu zakres dla tej funkcji został zawężony następująco:
\[x_i, y_i \in [-10, 10] \] \\

Posiada ona następujące globalne minimum:
\[ f(0, 0) = 0 \] \\

Funkcję tę można zaprezentować na wykresie 2 wymiarowym, co prezentują następujące obrazki:\\
\begin{figure}[H]
	\begin{center}
		\includegraphics[height=7cm]{quadraticFunction1.png}\\
	\end{center}
	\caption{Funkcja kwadratowa dwóch parametrów}
\end{figure}
\begin{figure}[H]
	\begin{center}
		\includegraphics[height=7cm]{quadraticFunction2.png}\\
	\end{center}
	\caption{Funkcja kwadratowa dwóch parametrów}
\end{figure}


	\subsection{Funkcja Rastrigina}
	Funkcja Rastrigina jest funkcją ciągłą, skalowalną i multimodalną. Dzięki posiadaniu wielu minimum lokalnych funkcja ta jest często stosowana w testowaniu algorytmów optymalizacyjnych. Przyjmuje ona następującą postać:

\[f(x) = An + \sum_{i=1}^{n} [x_i^2 - A \cos{(2 \pi x_i)}] \]

gdzie: \\
A = 10, \\
n = ilo\si ć wymiarów \\ \newLine

Zakres warto\si ci dla tej funkcji znajdziemy w przedziale:
\[x_i \in [-5.12, 5.12] \] \\

Posiada ona następujące globalne minimum:
\[ f(0,...,0) = 0 \] \\

By ujrzeć jej niektóre wła\si ciwo\si ci zaprezentowałem jej wykres w 2 wymiarach na poniższych obrazkach:\\
\begin{figure}[H]
	\begin{center}
		\includegraphics[height=7cm]{rastriginFunction1.png}\\
	\end{center}
	\caption{Funkcja Rastrigina o 2 wymiarach}
\end{figure}
\begin{figure}[H]
	\begin{center}
		\includegraphics[height=7cm]{rastriginFunction2.png}\\
	\end{center}
	\caption{Funkcja Rastrigina o 2 wymiarach}
\end{figure}



	\subsection{Funkcja Rosenbrocka}
Funkcja ta jest funkcją ciągłą, skalowalną i jednomodalną.

\[f(x) = \sum_{i=1}^{n-1} [100(x_{i+1} - x_i^2)^2 + (1- x_i)^2 ]\] \\


Funkcja ta przyjmuje wyłącznie warto\si ci nieujemnie. Na potrzeby projektu warto\si ci argumentów dla tej funkcji zostały zawężone do poniższego zakresu:
\[x_i \in [-10, 10] \] \\

Posiada ona następujące globalne minimum:
\[ f(1,...,1) = 0 \] \\

Poniższe wykresy prezentują jej wygląd w zadanym zakresie:\\
\begin{figure}[H]
	\begin{center}
		\includegraphics[height=7cm]{rosenbrockFunction1.png}\\
	\end{center}
	\caption{Funkcja Rosenbrocka o 2 wymiarach}
\end{figure}
\begin{figure}[H]
	\begin{center}
		\includegraphics[height=7cm]{rosenbrockFunction2.png}\\
	\end{center}
	\caption{Funkcja Rosenbrocka o 2 wymiarach}
\end{figure}

\section{Dobór parametrów}
Pomimo, iż algorytmy heurystyczne są dobrym wyborem wszędzie tam, gdzie ważny jest czas znalezienia rozwiązania, to przed skorzystaniem z danego algorytmu jeste\si my zmuszeni ustawić parametry algorytmu w taki sposób, by wynik był dostatecznie dokładny, a algorytm nie wykonywał niepotrzebnie obliczeń, zwłaszcza gdy większa dokładno\si ć nie jest nam potrzebna lub nie będzie stanowić większej różnicy w stosunku do już znalezionego wyniku. Dodatkową trudno\si ć stanowi ilo\si ć parametrów oraz to, iż każdy z nich może wpływać w inny sposób na złożono\si ć obliczeniową oraz wynik, oraz parametry mogą być od siebie zależne. W opracowaniach naukowych rzadko kiedy można znaleźć wytyczne co do sposobu znalezienia odpowiednich parametrów do konkretnych problemów. \\

Starając się trzymać zasad dotyczących tworzenia dobrego algorytmu heurystycznego, przyjąłem kilka założeń, a następnie sukcesywnie poszukiwałem odpowiednich warto\si ci dla parametrów by (\si redni, 10-krotne powtórzenia) wynik był jak najlepszy, starając się zawężać zakres z czasem. Kiedy (\si rednie) wyniki były już zadowalające, sprawdzałem jako\si ć dobranych parametrów wykonując 100 razy algorytm z takimi samymi parametrami dla tego samego problemu, uzysując w prosty sposób procentową jako\si ć algorytmu. W podsekcji "Dobieranie parametrów dla funkcji 5 wymiarowej funkcji Rastrigina" tabele przedstawią stopniowe doj\si cie do parametrów dających zadowalające wyniki, a następnie jako\si ć tych parametrów dla danego problemu. Parametry dla innych problemów zostały zbadane w taki sam sposób i w tych sekcjach zostanie wspomniany jedynie wynik.

	\subsection{Dobieranie parametrów dla funkcji 5 wymiarowej funkcji Rastrigina}

Przed rozpoczęciem testów przyjęto dwa założenia:
\begin{enumerate}
	\item Końcowa temperatura została ustawiona na stałą warto\si ć równą 0.001,
	\item Stopień chłodzenia temperatury został ustawiony na 0.99.
\end{enumerate}

\renewcommand\arraystretch{1.333}


\begin{tabularx}{\textwidth}{ | >{\rownum}c|X|X|X|} 
\hline
& \textbf{ T0} & \textbf{ Iteracje} &\textbf{ Rozwiązanie}\\ \hline
& 500 & 300 & 1,09718505292704 \\ \hline 
& 500 & 400 & 1,29513703551228 \\ \hline 
& 100 & 500 & 1,29579688714116 \\ \hline 
& 300 & 400 & 1,39516546905179 \\ \hline 
& 100 & 400 & 1,59400704106305 \\ \hline 
& 400 & 200 & 1,59460563439097 \\ \hline 
& 400 & 400 & 1,59473245683331 \\ \hline 
& 300 & 500 & 1,69368418646326 \\ \hline 
& 100 & 200 & 1,69430049805511 \\ \hline 
& 400 & 500 & 1,69488738622795 \\ \hline 
& 200 & 500 & 1,79356181491557 \\ \hline 
& 500 & 200 & 1,89311831496136 \\ \hline 
& 300 & 100 & 1,89379034166828 \\ \hline 
& 200 & 200 & 1,89405233085409 \\ \hline 
& 400 & 300 & 1,99208845088035 \\ \hline 
& 500 & 500 & 1,99250333315708 \\ \hline 
& 300 & 300 & 1,99272286399703 \\ \hline 
& 200 & 400 & 2,09157445324622 \\ \hline 
& 100 & 300 & 2,19168206783873 \\ \hline 
& 300 & 200 & 2,19179918169512 \\ \hline 
& 500 & 100 & 2,19209302230291 \\ \hline 
& 200 & 100 & 2,19237079270076 \\ \hline 
& 400 & 100 & 2,19237108578013 \\ \hline 
& 200 & 300 & 2,29096718181202 \\ \hline 
& 100 & 100 & 2,98943206230488 \\ \hline 
\end{tabularx}

Badanie zawarte w tabeli 1 pokazuje, iz parametry na takim poziomie nie maja az tak duzego znaczenia, jednak mozna zauwazyc, iz wieksza wartosc parametrow prowadzi do nieco lepszych srednich wynikow.
Badanie skutecznosci dla najwyzszych parametrow (czyli 500 i 500) wynioslo 12\%, co jest bardzo slabym wynikiem.


\begin{tabularx}{\textwidth}{ | >{\rownum}c|X|X|X|} 
                 \hline
                 & \textbf{ T0}
                 & \textbf{ Iteracje}
                 &\textbf{ Rozwiązanie}\\ \hline
& 8000 & 6000 & 0,400456362305732 \\ \hline 
& 10000 & 4000 & 0,400942457997776 \\ \hline 
& 4000 & 8000 & 0,499468816488703 \\ \hline 
& 4000 & 10000 & 0,499496708877431 \\ \hline 
& 6000 & 6000 & 0,499881155483216 \\ \hline 
& 8000 & 8000 & 0,500190776858622 \\ \hline 
& 8000 & 10000 & 0,500669723808464 \\ \hline 
& 2000 & 6000 & 0,500802825547873 \\ \hline 
& 2000 & 4000 & 0,59876018888673 \\ \hline 
& 2000 & 8000 & 0,598944859155338 \\ \hline 
& 10000 & 8000 & 0,69839967735632 \\ \hline 
& 6000 & 8000 & 0,698671405088412 \\ \hline 
& 10000 & 10000 & 0,698922421188404 \\ \hline 
& 4000 & 6000 & 0,797834332682657 \\ \hline 
& 6000 & 10000 & 0,798499451192562 \\ \hline 
& 4000 & 4000 & 0,798596850008959 \\ \hline 
& 2000 & 10000 & 0,798896772938739 \\ \hline 
& 8000 & 4000 & 0,79899550599823 \\ \hline 
& 10000 & 6000 & 0,898177793387154 \\ \hline 
& 6000 & 4000 & 0,898733662378301 \\ \hline 
& 4000 & 2000 & 0,996463628047726 \\ \hline 
& 8000 & 2000 & 0,997870530143807 \\ \hline 
& 6000 & 2000 & 1,09703634442711 \\ \hline 
& 2000 & 2000 & 1,09731890136792 \\ \hline 
& 10000 & 2000 & 1,39564290923108 \\ \hline 
\end{tabularx}
10k i 10k
46\%\
5k - 10k -> 46\%
5k i 20k -> 58\%\
5k i 30k ->  72\%

\scalebox{0.5}{
\begin{tabularx}{\textwidth}{ | >{\rownum}c|X|X|X|}
  \hline
    & \textbf{T0} & \textbf{Iteracje} &\textbf{Rozwiązanie}\\ \hline
    & aaaaaaa aaaaaaa & aaaaaaa aaaaaaa  & oko\\ \hline
    & aaaaaaa aaaaaaa &aaaaaaa & oko \\ \hline
    & aaaaaaa & aaaaaaaaaaaaaa & oko \\ \hline
 & aaaaaaa & aaaaaaaaaaaaaa & oko \\ \hline
 & aaaaaaa & aaaaaaaaaaaaaa & oko \\ \hline
 & aaaaaaa & aaaaaaaaaaaaaa & oko \\ \hline
 & aaaaaaa & aaaaaaaaaaaaaa & oko \\ \hline
 & aaaaaaa & aaaaaaaaaaaaaa & oko \\ \hline
 & aaaaaaa & aaaaaaaaaaaaaa & oko \\ \hline
 & aaaaaaa & aaaaaaaaaaaaaa & oko \\ \hline
 & aaaaaaa & aaaaaaaaaaaaaa & oko \\ \hline
 & aaaaaaa & aaaaaaaaaaaaaa & oko \\ \hline
   & aaaaaaa aaaaaaa & aaaaaaa aaaaaaa  & oko\\ \hline
    & aaaaaaa aaaaaaa &aaaaaaa & oko \\ \hline
    & aaaaaaa & aaaaaaaaaaaaaa & oko \\ \hline
 & aaaaaaa & aaaaaaaaaaaaaa & oko \\ \hline
 & aaaaaaa & aaaaaaaaaaaaaa & oko \\ \hline
 & aaaaaaa & aaaaaaaaaaaaaa & oko \\ \hline
 & aaaaaaa & aaaaaaaaaaaaaa & oko \\ \hline
 & aaaaaaa & aaaaaaaaaaaaaa & oko \\ \hline
 & aaaaaaa & aaaaaaaaaaaaaa & oko \\ \hline
 & aaaaaaa & aaaaaaaaaaaaaa & oko \\ \hline
 & aaaaaaa & aaaaaaaaaaaaaa & oko \\ \hline
 & aaaaaaa & aaaaaaaaaaaaaa & oko \\ \hline
 & aaaaaaa & aaaaaaaaaaaaaa & oko \\ \hline
 & aaaaaaa & aaaaaaaaaaaaaa & oko \\ \hline
 & aaaaaaa & aaaaaaaaaaaaaa & oko \\ \hline
 & aaaaaaa & aaaaaaaaaaaaaa & oko \\ \hline
 & aaaaaaa & aaaaaaaaaaaaaa & oko \\ \hline
 & aaaaaaa & aaaaaaaaaaaaaa & oko \\ \hline
 & aaaaaaa & aaaaaaaaaaaaaa & oko \\ \hline
 & aaaaaaa & aaaaaaaaaaaaaa & oko \\ \hline
 & aaaaaaa & aaaaaaaaaaaaaa & oko \\ \hline

\end{tabularx}
}







\begin{tabularx}{\textwidth}{ | >{\rownum}c|X|X|X|}
  \hline
    & \textbf{T0} & \textbf{Iteracje} &\textbf{Rozwiązanie}\\ \hline
    & aaaaaaa aaaaaaa & aaaaaaa aaaaaaa  & oko\\ \hline
    & aaaaaaa aaaaaaa &aaaaaaa & oko \\ \hline
    & aaaaaaa & aaaaaaaaaaaaaa & oko \\ \hline
 & aaaaaaa & aaaaaaaaaaaaaa & oko \\ \hline
 & aaaaaaa & aaaaaaaaaaaaaa & oko \\ \hline
 & aaaaaaa & aaaaaaaaaaaaaa & oko \\ \hline
 & aaaaaaa & aaaaaaaaaaaaaa & oko \\ \hline
 & aaaaaaa & aaaaaaaaaaaaaa & oko \\ \hline
 & aaaaaaa & aaaaaaaaaaaaaa & oko \\ \hline
 & aaaaaaa & aaaaaaaaaaaaaa & oko \\ \hline
 & aaaaaaa & aaaaaaaaaaaaaa & oko \\ \hline
 & aaaaaaa & aaaaaaaaaaaaaa & oko \\ \hline

\end{tabularx}


\begin{tabularx}{\textwidth}{ | >{\rownum}c|X|X|X|}
  \hline
    & \textbf{T0} & \textbf{Iteracje} &\textbf{Rozwiązanie}\\ \hline
    & aaaaaaa aaaaaaa & aaaaaaa aaaaaaa  & oko\\ \hline
    & aaaaaaa aaaaaaa &aaaaaaa & oko \\ \hline
    & aaaaaaa & aaaaaaaaaaaaaa & oko \\ \hline
 & aaaaaaa & aaaaaaaaaaaaaa & oko \\ \hline
 & aaaaaaa & aaaaaaaaaaaaaa & oko \\ \hline
 & aaaaaaa & aaaaaaaaaaaaaa & oko \\ \hline
 & aaaaaaa & aaaaaaaaaaaaaa & oko \\ \hline
 & aaaaaaa & aaaaaaaaaaaaaa & oko \\ \hline
 & aaaaaaa & aaaaaaaaaaaaaa & oko \\ \hline
 & aaaaaaa & aaaaaaaaaaaaaa & oko \\ \hline
 & aaaaaaa & aaaaaaaaaaaaaa & oko \\ \hline
 & aaaaaaa & aaaaaaaaaaaaaa & oko \\ \hline

\end{tabularx}

Ostatecznie wybrane parametry dla tego problemu:\\
pocz temp\\
konc temp\\
cooling\\
iteracje\\

Badanie skutecznosci \\






skutecznosc:\\






	\subsection{Dobieranie parametrów dla funkcji 3 wymiarowej funkcji Rastrigina}
Ostatecznie wybrane parametry dla tego problemu:\\
pocz temp\\
konc temp\\
cooling\\
iteracje\\






skutecznosc:\\


	\subsection{Parametry dobrane dla funkcji kwadratowej dwóch parametrów}
Ostatecznie wybrane parametry dla tego problemu:\\
pocz temp\\
konc temp\\
cooling\\
iteracje\\






skutecznosc:\\

	\subsection{Parametry dobrane dla funkcji Rosenbrocka}
Ostatecznie wybrane parametry dla tego problemu:\\
pocz temp\\
konc temp\\
cooling\\
iteracje\\






skutecznosc:\\

\section{Implementacja}
\section{Dostosowanie algorytmów do funkcji testowej zadań odwrotnych}
\section{Narzędzia i technologie}
	\subsection{Metodyka pracy}
	\subsubsection{System kontroli wersji}
	System kontroli wersji posiada wiele zalet, jak m.in.: bezpieczeństwo, możliwo\si ć pracy w kilku miejscach/urządzeniach nad tym samym problemem, łatwą możliwo\si ć przywrócenia poprzedniej wersji, czy wreszcie, inspekcję jako\si ci i poprawno\si ci kodu. \\
W moim projekcie skorzystałem z systemu kontroli Git, a repozytorium można znaleźć na portalu github.com. 
	\subsubsection{Github Project Management}
Pomimo, iż praca w pojedynkę nie wymagała ode mnie zarządzania projektem i konieczno\si ci organizacji pracy, zdecydowałem się na użycie narzędzia pozwalającego na taką pracę. Podzielenie projektu na mniejsze zadania pozwoliło mi wydzielić poszczególne i odrębne sektory pracy, widzieć postępujący progres i łatwo odnaleźć się w aktualnie wykonywanym zadaniu. W tym celu skorzystałem z Github Project Management, który pozwala na proste zarządzanie zadaniami.
	\subsubsection{\SI rodowisko programistyczne}
Do implementacji projektu użyłem \si rodowiska Microsoft Visual Studio Community 2017, które to zostało stworzone przez firmę Microsoft i pozwala na programowanie konsolowe oraz z graficznym interfejsem użytkownika (zarówno aplikacje desktopowe, jak i strony internetowe).  \\
Dobra znajomo\si ć i przejrzysto\si ć tego \si rodowiska programistycznego pozwoliła mi skupić się na rozwiązywaniu problemu, omijając problem zapoznawania się z nowym narzędziem.
	\subsubsection{Mathematica}
	Mathematica jest programem opartym na systemie obliczeń symbolicznych oraz numerycznych. Program ten jest do\si ć popularny w\si ród naukowców ze względu na wiele zalet, jak np. wydajno\si ć czy rozpięte możliwo\si ci wizualizacji danych. Mathematica jest programem komercyjnym, dlatego stworzenie wykresów do tego projektu oparłem na licencji wydziału Matematyki Stosowanej.
	
	\subsection{Użyte technologie}
	\subsubsection{C\#}
Język programowania C\# należy do obiektowych języków programowania, którego koncepcja opiera się na tworzeniu klas, które poprzez swoją zawarto\si ć (m.in. wła\si ciwo\si ci czy metody) mogą być reprezentowane poprzez obiekty i każde operacje są wykonywane poprzez nie. W projekcie korzystam z języka C\# w wersji 7.0, która w momencie rozpoczęcia pracy była aktualna. Dobra znajomo\si ć tego języka pozwoliła mi nie zważać na problemy w znajomo\si ci składni czy funkcji i skupić się bezpo\si rednio na implementacji algorytmów, dobraniu odpowiednich parametrów dla poszczególnych funkcji testowych oraz lepszym przetestowaniu całej funkcjonalno\si ci.

\subsubsection{Wolfram Language}
Język ten służy głównie do programowania obliczeń matematycznych i programowania funkcjonalnego w programie Mathematica. Język ten, wraz z oprogramowaniem Mathematica, pozwalają m.in. na: operacje na macierzach, rozwiązywanie równań różniczkowych czy prezentowanie danych za pomocą wykresów. Z tej ostatniej funkcjonalno\si ci skorzystałem tworząc wykresy funkcji testowych.

\section{Podsumowanie}
	\subsection{Dalsze kierunki rozwoju}
	\subsection{Źródła}
%funkcje testowe
https://www.ncbi.nlm.nih.gov/pmc/articles/PMC4538776/

%wstep i opis problemu[?]
F. Rothlauf, Design of Modern Heuristics, Natural Computing Series,
DOI 10.1007/978-3-540-72962-4 2, © Springer-Verlag Berlin Heidelberg 2011

%wstep i opis problemu
R. Mart'ı and G. Reinelt, The Linear Ordering Problem, Exact and Heuristic Methods
in Combinatorial Optimization 175, DOI: 10.1007/978-3-642-16729-4 2,
c Springer-Verlag Berlin Heidelberg 2011


----------------------------------------------------------------------------------------------------------------------
%% UWaga na \newlineTekst oraz \newlineSpis. Można też użyć \newline, działa jak \newlineSpis\newlineTekst
\section[Tytuł drugiego rozdziału. Bardzo długi \ldots]
        {Tytuł drugiego rozdziału. \newlineTekst Bardzo długi tytuł. \newlineTekst
          Jego \newlineSpis formatowanie jest trudniejsze}



\dodatek{Mój specjalny dodatek}

Tu treść dodatku. Zwróćmy uwagę na sposób numerowania dodatku, 
możliwa jest zmiana numerowania, patrz wyjaśnienia.
          
%% to wpisuje się do spisu treści, ale bez numeru rozdziału,
%% można też używać \dodatek{Tytuł}, który jest numerowany, ale inaczej niż rozdziały.
\dodatkowo{Rysunki}

Tu rysunki

\dodatkowo{Programy}

Tu programy

\begin{verbatim}
#include <stdio.h>

int main()
{
   printf("Hello world\n");
}
\end{verbatim}

\noindent
Oraz 

\bigskip

\vrule\hspace{10pt}\begin{minipage}{10cm}
\begin{verbatim*}
<?php
   echo "test=$test";
?>
\end{verbatim*}
\end{minipage}

\begin{tw}
Twierdzenie Twierdzenie Twierdzenie Twierdzenie Twierdzenie 
\end{tw}
\begin{thebibliography}{12}

\bibitem{PozNazwa1} Jakaś pozycja literatury
\bibitem{InnPoz} Jakaś pozycja literatury

\end{thebibliography}
\end{document}
