\documentclass[12pt]{report}

%Packages
%-------------------------------------------------------------------------------
	\usepackage[T1]{fontenc}
	\usepackage[a4paper, left=1in, top=0.1in]{geometry}%Margins
	\usepackage{polski}
	\usepackage[utf8]{inputenc}
	\usepackage[polish]{babel}
	\usepackage{graphicx}
	\usepackage{fix-cm}
	\usepackage{amsmath}%Math equations
	\usepackage{tocloft}%Dots in table of contents
	\usepackage{indentfirst}%Indent at the begining, after section/subsection
	%\usepackage{fancyhdr}%Headers/footers
%-------------------------------------------------------------------------------
%end Packages



%Headers and footers
%-------------------------------------------------------------------------------
%\pagestyle{fancy}
%\chead{\Huge{\uppercase{POLITECHNIKA ŚLĄSKA}}}
%\lfoot{Gliwice, rok akadem. 2018/2019}
%-------------------------------------------------------------------------------
%end Headers and footers



%Metadata
%-------------------------------------------------------------------------------
\title{Porównanie wybranych algorytmów heurystycznych w rozwiązywaniu zagadnień odwrotnych}
\author{Kamil Kryus}
%-------------------------------------------------------------------------------
%end Metadata

%Renew
%-------------------------------------------------------------------------------
\renewcommand \cftchapdotsep{4.5}%Dots in table of contents
%-------------------------------------------------------------------------------
%end Renew

%Custom commands
%-------------------------------------------------------------------------------
\newcommand{\hugeFontSize}{}
\newcommand{\newLine}{~\\}
\newcommand{\si}{ś}
\newcommand{\SI}{Ś}
%-------------------------------------------------------------------------------
%end Custom commands

\makeatletter
\begin{document}
	\begin{titlepage}
	%\thispagestyle{fancy}
	\pagenumbering{gobble}
		\begin{center}
			\newLine

			\vskip 1.5in
			\includegraphics[width=0.4\textwidth]{logo}\\ \newLine
			\huge{Wydział Matematyki Stosowanej} \\ \newLine

			PRACA INŻYNIERSKA \\ \newLine
		\end{center}
			\LARGE{Temat: \@title} \\ \newLine
			Dyplomant: \@author \\
			Kierunek studiów: Informatyka \\
			Specjalizacja: Programowanie Aplikacji Mobilnych \\ \newLine
			Opiekun pracy: dr inż. Adam Zielonka \\
	\end{titlepage}
	\newpage
%Oswiadczenia
\noindent {\Huge {\textbf{O\si wiadczenie kierującego projektem inżynierskim}}} \\ \newLine \\ \newLine
\normalsize
\indent  Potwierdzam, że niniejszy projekt został przygotowany pod moim kierunkiem i kwalifikuje się do przedstawienia go w postępowaniu o nadanie tytułu zawodowego: inżynier.\\ \newLine \\ \newLine

Data
\hfill
Podpis kierującego projektem\\ \newLine





\noindent {\Huge{\textbf{O\si wiadczenie autora}} \\ \newLine \\ \newLine
\normalsize
\indent \SI wiadomy odpowiedzialno\si ci karnej o\si wiadczam, że przedkładany projekt inżynierski na temat:
\@title
został napisany przez autra samodzielnie.

\noindent Jednocze\si nie o\si wiadczam, że ww. projekt: \\


\setlength{\leftskip}{1cm}

\noindent-nie narusza praw autorskich w rozumieniu ustawy z dnia 4 lutego 1994 roku o prawie autorskim i prawach pokrewnych (Dz. U. z 2000 r. Nr 80, poz. 904 z późn. zm.) oraz dóbr osobistych chronionych prawem cywilnym, a także nie zawiera danych i informacji, które uzyskałem w sposób niedozwolony, \\

\noindent-nie była wcze\si niej podstawą żadnej innej urzędowej procedury związanej z nadawaniem dyplomów wyższej uczelni lub tytułów zawodowych,\\ 

\noindent-nie zawiera fragmentów dokumentów kopiowanych z innych źródeł bez wyraźnego zaznaczenia i podania źródła.\\

\setlength{\leftskip}{0pt}
\noindent Podpis autora projektu:\\ \newLine

\noindent1. Kamil Kryus  \hfill nr albumu: 246591 \hfill $ \underset{podpis}{......................................}$ \\ \newLine \\ \newLine

\hfill Gliwice, dnia ...................
	\newpage
%end Oswiadczenia



	\thispagestyle{plain}
	\pagenumbering{arabic}
%Table of Contents
	\tableofcontents
%end Table of Contents
	\newpage
%Wprowadzenie
	\chapter{Wprowadzenie}
	
%end Wprowadzenie
%Opis problemu
	\chapter{Opis problemu}

%end Opis problemu
%Opis algorytmów
	\chapter{Opis algorytmów}
		\section{Algorytm symulowanego wyżarzania}
	Algorytm ten został stworzony wzorując się na zjawisku wyżarzania w metalurgii, które polega na nagrzaniu elementu stalowego do odpowiedniej temperatury, przetrzymaniu go w tej temperaturze przez pewien czas, a następnie powolnym jego schłodzeniu. Sam algorytm natomiast bazuje na metodach Monte-Carlo i w pewnym sensie może być rozważany jako algorytm iteracyjny.\\ \newLine
Główną istotą i zarazem zaletą tego algorytmu jest wykonywanie pewnych losowych przeskoków do sąsiednich rozwiązań, dzięki czemu jest w stanie uniknąć wpadania w lokalne minimum. Algorytm ten najczę\si ciej jest używany do rozwiązywania problemów kombinatorycznych, jak np. problemu komiwojażera.
			\subsection{Parametry}

\noindent \underline{Początkowa konfiguracja} \\ \newLine
\indent W tym kroku powinniśmy zainicjalizować naszą temperaturę wysoką wartością oraz znaleźć początkowe losowe rozwiązanie naszego problemu. 
\\ \newLine

\noindent \underline{Temperatura} \\ \newLine
\indent Temperatura jest zarówno czynnikiem iteracyjnym, jak i jest związana z funkcją prawdopodobieństwa zamiany gorszego rozwiązania z lepszym. Zatem zakres temperatury powinien być taki, aby na początku działania naszego algorytmu dawał wysoką możliwość zamian, a wraz z postępem iteracji te prawdopodobieństwo się zmniejszało i pod koniec było bliskie zeru.\\ \newLine


\noindent \underline{Końcowa temperatura} \\ \newLine
\indent Jest to bardzo mała wartość, która sugeruje, iż gdy wartość temperatury spadnie do tego poziomu, zakończył się proces wyżarzania i rozwiązanie zostało znalezione. Wartość ta powinna być na tyle mała, by temperatura będąc niewiele większa prowadziła do bardzo niskiego prawdopodobieństwa, a jednocze\si nie nie wymagało to zbyt dużej ilo\si ci iteracji. \\ \newLine


\noindent \underline{Powtarzanie pewną ilość razy dla zadanej temperatury} \\ \newLine
\indent Wartość ta powinna być z góry ustalona i powinna dać nam możliwość sprawdzenia wielu sąsiadów obecnego rozwiązania, równocześnie nie powodując zbyt dużego obciążenia dla algorytmu.\\ \newLine


\noindent \underline{Znajdowanie losowego sąsiada poprzedniego rozwiązania} \\ \newLine
\indent Funkcja ta powinna nam pozwalać przejrzeć jak najszerszy zakres rozwiązań, a jednocze\si nie pozwolić na przeszukiwanie coraz to bliższych sąsiadów obecnie najlepszego rozwiązania, zatem warto uzależnić tą funkcję od stopnia ukończenia globalnych iteracji.\\ \newLine


\noindent \underline{Funkcja kosztu} \\ \newLine
\indent Poprzez funkcję kosztu rozumiemy różnicę pomiędzy obecnie najlepszym rozwiązaniem, a nowym. W tym wypadku nowe rozwiązanie zawsze jest gorsze (a tym samym przy poszukiwaniu minimum globalnego posiada warto\si ć większą), to w tym wypadku warto\si ć jest ujemna.\\ \newLine


\noindent \underline{Prawdopodobieństwo zamiany P} \\ \newLine
\indent Prawdopodobieństwo jest potrzebne do decyzji czy zamienić nasze nowe i gorsze rozwiązanie z wcze\si niejszym, lepszym. 

Prawdopodobieństwo zależy od różnicy pomiędzy starą wartością, a nową, zwaną tutaj dystansem, oraz obecnej temperatury. Prawdopodobieństwo zatem można przedstawić w następujący sposób:
\[P = e^\frac{\Delta E}{T} \]
gdzie:\\
\begin{flalign}
& \Delta E - funkcja\ kosztu& 
\end{flalign}
T - obecna warto\si ć temperatury\\

Prawdopodobieństwo to wraz ze spadkiem warto\si ci funkcji kosztu maleje (gdyż jest zawsze ujemne), natomiast wyższa warto\si ć temperatury zwiększa prawdopodobieństwo. Decydując o tym czy powinniśmy zamienić nasze gorsze rozwiązanie z lepszym powinniśmy porównać z wartością losową zawierającą się w zakresie [0,1].\\ \newLine


\noindent \underline{Zmniejszanie temperatury} \\ \newLine
\indent Szybkość zmniejszania nie powinna być zbyt duża, aby pozwolić algorytmowi na sprawdzenie jak największego zakresu możliwych rozwiązań, a jednocześnie niezbyt wolna, gdyż może to spowodować zbyt wolny spadek prawdopodobieństwa, a tym samym finalnie pozwolić na pozostanie w gorszym rozwiązaniu. W większości opracowań można spotkać ten proces, jako mnożnik temperatury w zakresie [0.8;0.99].\\ \newLine



			\subsection{Kroki algorytmu}
Algorytm ten można również przedstawić za pomocą listy kroków:

\begin{enumerate}
	\item Zainicjalizuj początkową konfigurację
	\item Dopóki temperatura > minimum, powtarzaj:
	\begin{enumerate}
		\item Powtórz pewną ilość razy dla danej temperatury
		\begin{enumerate}
			\item Znajdź losowo sąsiada poprzedniego rozwiązania
			\item Sprawdź czy rozwiązanie jest lepsze od poprzedniego (funkcja kosztu)
			\begin{enumerate}
				\item Jeżeli jest, zamień rozwiązania
				\item Jeżeli nie jest, zamień rozwiązania z pewnym prawdopodobieństwem P
			\end{enumerate}
		\end{enumerate}
		\item Zmniejsz temperaturę
	\end{enumerate}
\end{enumerate}

%end Opis algorytmów
%Cel
	\chapter{Cel}
	Przetestowanie algorytmu wyżarzania, bla bla
%end Cel
%Funkcje testowe
	\chapter{Funkcje testowe}
		\section{Funkcja kwadratowa dwóch parametrów}
Funkcja ta ma następującą postać:

\[f(x, y) = x^2 + y^2 \]

Co\si o wykresie:
<tutaj jakis wykres >

Funkcja ta przyjmuje jedynie warto\si ci dodatnie, zakres dla potrzeb projektu został zawężony do [-10, 10], natomiast minimum globalne posiada w punkcie [0, 0] i wynosi 0. 



		\section{Funkcja Rastrigina}
Funkcja Rastrigina jest funkcją ciągłą, skalowalną i multimodalną. Dzięki posiadaniu wielu minimum lokalnych funkcja ta jest często stosowana w testowaniu algorytmów optymalizacyjnych. 

\[f(x) = An + \sum_{i=1}^{n} [x_i^2 - A \cos{(2 \pi x_i)}] \]

gdzie: \\
A = ilo\si ć wymiarów \\ \newLine
Parametry funkcji możemy znaleźć w zakresie [-5.12; 5.12], minimum globalne natomiast w punkcie [0,...,0] i wynosi 0.

		\section{Funkcja Rosenbrocka}
Rosenbrock's valley function is known as the second function of De Jong. This test function is continuous, scalable, naturally nonseparable, nonconvex, and unimodal.


\[f(x) = \sum_{i=1}^{n-1} [100(x_{i+1} - x_i^2)^2 + (1- x_i)^2 ]\]

%end Funkcje testowe
%Dobór parametrów
	\chapter{Dobór parametrów}

%end Dobór parametrów
%Implementacja
	\chapter{Implementacja}

%end Implementacja
%Dostosowanie algorytmów do funkcji testowej zadań odwrotnych
	\chapter{Dostosowanie algorytmów do funkcji testowej zadań odwrotnych}

%end Dostoswanie algorytmów do funkcji testowej zadań odwrotnych
%Narzędzia i technologie
	\chapter{Narzędzia i technologie}
Aby zapewnić bezpieczeństwo oraz możliwo\si ć pracy w kilku miejscach nad jednym problemem, zastosowałem kilka rozwiązań.
		\section{Użyte narzędzia}
\subsection{System kontroli wersji}
Git
\subsection{Podział pracy na mniejsze zadania}
Git

		\section{Użyte technologie}
.NET Framework
%end Narzędzia i technologie
%Podsumowanie
	\chapter{Podsumowanie}
%end Podsumowanie
		\section{Dalsze kierunki rozwoju}
	Foobar

%end Dalsze kierunki rozwoju
%Źródła
		\section{Źródła}
https://www.ncbi.nlm.nih.gov/pmc/articles/PMC4538776/
%end Źródła
\end{document}